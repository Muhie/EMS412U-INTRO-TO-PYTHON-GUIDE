\documentclass{article}
% -- Loading the code block package:
\usepackage{listings}
% -- Basic formatting
\usepackage[utf8]{inputenc}
\usepackage[english]{babel}
\usepackage{times}
\setlength{\parindent}{10pt}
\usepackage{minted}
\usepackage{indentfirst}
\usepackage{hyperref}
% -- Defining colors:
\usepackage[dvipsnames]{xcolor}
\definecolor{codegreen}{rgb}{0,0.6,0}
\definecolor{codegray}{rgb}{0.5,0.5,0.5}
\definecolor{codepurple}{rgb}{0.58,0,0.82}
\definecolor{backcolour}{rgb}{0.95,0.95,0.92}
% Definig a custom style:
\lstdefinestyle{mystyle}{
	backgroundcolor=\color{backcolour},   
	commentstyle=\color{codepurple},
	keywordstyle=\color{NavyBlue},
	numberstyle=\tiny\color{codegray},
	stringstyle=\color{codepurple},
	basicstyle=\ttfamily\footnotesize\bfseries,
	breakatwhitespace=false,         
	breaklines=true,                 
	captionpos=t,                    
	keepspaces=true,                 
	numbers=left,                    
	numbersep=-5pt,                  
	showspaces=false,                
	showstringspaces=false,
	showtabs=false,                  
	tabsize=2
}
\usepackage[fencedCode]{markdown}
% -- Setting up the custom style:
\lstset{style=mystyle}
\title{An Engineer's Beginners Guide To Python (from a students perspective)}
\author{Muhie Al Haimus}

\begin{document}
	\maketitle
	\section{Introduction}
	This document could be thought as the one stop shop to succeed in EMS412U python side of the module.  
	 \section{Week 0: The laying the foundation}
	 Let's start with the most basic python program and explain some key concepts. Below is a "code block", this contains the actual raw python code.
	 
	\begin{lstlisting}[language=Python]
	# Author: Muhie
	# Date: 21/05/24
	# Explaination of the program: Outputs a Hello World! to the screen
	"""I can
	write a comment 
	on multiple lines"""
	print("Hello World!")
	\end{lstlisting}
	\begin{lstlisting}[language=Bash]
		Output:
		Hello World!
	\end{lstlisting}
	\emph{Lines 1-3}: In every program you write you MUST always start with:
	\begin{itemize}
		\item{Your name}
		\item{Today's date}
		\item{A brief explanation of the program}
	\end{itemize}
	As you can see in the example above before writing my name, date and explanation I use a hashtag (\#) symbol, this denotes a comment. A comment is completely separate to the main program, comments do not effect the functionality of the program. One more thing to note is that if you need a comment over multiple lines you can use three quotation marks: three before the start of the comment and then three after the comment, as seen above with \emph{lines 4-6.}
	\vspace{2mm}
	\emph{Line 7}: This is where the actual python code starts the keyword  
	
	
	 Syntax - these are mainly keywords (and symbols like a hashtag \# or a combination of symbols and keywords like print("") ).
	 
	 
	 \section{KEY CONCEPTS YOU'LL NEED FOR WEEKS 1-12}
	 \subsection{Week 1: Variables}
	 A variable is simply a value that can change. A variable can have many different types below is the most common types that you will use throughout the module.
	 \begin{lstlisting}[language=python]
	 # Author: Muhie
	 # Date: 22/05/24
	 """Explaination of the program: A simple program showing how variables can be assigned in python"""
	 
	 a = 5 # integer
	 b = "hello" # string
	 c = True # boolean
	 d = 3.14 # float
	 \end{lstlisting}
	 
	 \subsection{Week 2: Selection and Iteration}
	 	 \begin{lstlisting}[language=python]
	 	# Author: Muhie
	 	# Date: 22/05/24
	 	"""Explaination of the program: A simple program showing what a for loop does and what an if statement is"""
	 	start = 0 
	 	stop = 10
	 	for i in range(start,stop):
	 		print(i)
	 		
	 	a = 5
	 	if a > 5:
	 		print("a is greater than 5!")
	 	elif a == 5:
	 		print("a is equal to 5!")
	 	else:
	 		print("a is not equal to 5 or greater than 5")
	 	
	 	
	 \end{lstlisting}
	 
	 
	 \subsection{Week 3: }
	 
	 
	 \subsection{Key syntax}
	\begin{lstlisting}[language=python]
	 	import (your library name here) # this imports a library, a library
	 \end{lstlisting}
	 
	
	\section{I NEED HELP WHAT DO I DO!}
	\href{https://www.w3schools.com/python/python_syntax.asp}{A good reference guide to python syntax and concepts}
	
\end{document}